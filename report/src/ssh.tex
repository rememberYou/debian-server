\section{SSH}
\label{sec:ssh}

Le \textbf{SSH}, ou \textit{\textbf{Secure Shell}}, est un protocole de
communication sécurisé. Il impose un échange de clés de chiffrement en début de
connexion.


\subsection{Choix de l'authentification}
\label{subsec:choix-authentification}

Il y a plusieurs façons de s'authentifier sur le serveur via SSH. \\
Les deux plus utilisées sont :
\begin{itemize}
    \item l'authentification par mot de passe;
    \item l'authentification par clés publique et privée du client. \\
\end{itemize}

Nous avons décidé de mettre en place une \textbf{identification automatique par
clés}. Ainsi on évite d'entrer le mot de passe à chaque connexion à distance. \\
Cette méthode est plus complexe à mettre en place, mais elle surtout plus
pratique. \\

On remarque rapidement son utilité si on se connecte fréquemment au serveur, car
plus aucun mot de passe n'est demandé.


\subsection{Mise en place}
\label{subsec:mise-en-place}
