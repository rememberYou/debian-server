\chapter*{Gestion}
\label{ch:gestion}


\section{Sauvegardes}
\label{subsec:sauvegardes}

Dans le milieu de l'entreprise, deux types de sauvegarde sont utilisées :
incrémentielle et différentielle.

La méthode de sauvegarde choisie est la différentielle, afin de restaurer les
données plus rapidement par rapport à la sauvegarde incrémentielle. De plus,
cette méthode est plus fiable, car seule la sauvegarde complète est nécessaire
pour reconstituer les données sauvegardées.

Il est à remarqué que la sauvegarde incrémentielle est plus économe en terme de
stockage.

Pour terminer, nous prévoyons d'effectuer une sauvegarde complète
\textit{(différentielle)} une fois toutes les semaines.

\newpage


\subsection{Horaire de sauvegarde}
\label{subsec:horaire-sauvegarde}

Nous avons décidé de mettre en place deux types de sauvegarde : une sauvegarde
incrémentielle tous les jours à deux heures du matin, et une sauvegarde complète
tous les dimanches à nouveau à deux heures du matin. Chaque sauvegarde se fera
sur un disque dur externe. \\

L'heure a été choisie de cette manière, car la machine, étant un serveur, n'est
jamais arrêtée et, à ce moment, personne ne devrait travailler dessus. \\
Le choix du dimanche en découle : en plus d'être à une plage horaire sans
travailleur \textit{(ou très peu)}, le dimanche est un jour de congé pour la
majorité du monde, ce qui aura un impact mineur sur les performances du serveur.


\subsection{Mise en \oe{}uvre}
\label{subsec:mise-en-oeuvre)}

Les sauvegardes ont été automatisées à l'aide de \textit{cron} et configurées
comme suit :
\begin{itemize}

    \item[$\bullet$] création du script à exécuter pour les deux types de
    sauvegarde : sauvegarde des différents dossiers systèmes
    \textit{/dev, /proc, /sys, /tmp, /srv/share, etc.)};
    \item[$\bullet$] création du script de restauration des dossiers systèmes
    sauvegardés;
    \item[$\bullet$] ajout des horaires d'exécution des scripts à l'aide de
    \textit{crontab};
    \item[$\bullet$] redémarrage de cron;
    \item[$\bullet$] ajout d'une sauvegarde bootable. \\

\end{itemize}

\underline{Remarque :} une documentation accompagne chaque type de sauvegarde.


%%% Local Variables:
%%% mode: latex
%%% TeX-master: t
%%% End:
