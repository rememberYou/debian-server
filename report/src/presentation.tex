\section{Présentation générale du projet}
\label{sec:pres-gener-du}


\subsection{Introduction}
\label{subsec:introduction}

Dans le cadre de ce projet, il nous a été demandé d'administrer un serveur sous
Linux. \\
Le choix de la distribution ainsi que la gestion des sauvegardes est libre et
devra être justifié. \\

Le serveur devra contenir :

\begin{itemize}
    \item un partage \textbf{NFS} qui permettra aux utilisateurs du réseaux local d'y stocker des fichiers;
    \item un partage \textbf{Samba} permettra aux utilisateurs Windows d'accéder à ce même partage;
    \item un serveur \textbf{Web}, \textbf{FTP}, \textbf{MySQL} et \textbf{DNS} qui permettra un hébergement multi-utilisateurs;
    \begin{itemize}
        \item[$\bullet$] le FTP permettra à chaque utilisateur d'accéder à son dossier Web;
        \item[$\bullet$] il faudra créer une zone dans le DNS pour nos sites;
        \item[$\bullet$] le DNS fera également office de DNS cache pour le réseau local;
    \end{itemize}
    \item un \textbf{serveur de temps} pour que les machines du
    réseau local puissent se synchroniser;
    \item une connexion en \textbf{SSH} au serveur.
\end{itemize}


\newpage


\subsection{Déontologie}
\label{subsec:déontologie}

En tant qu'administrateurs du serveur, nous serons tenus de suivre de nombreuses
règles telles que :

\begin{itemize}
    \item la documentation des actions entreprises sur le serveur;
    \item l'automatisation des installations et configurations au travers de scripts;
    \item la sécurité : mise en place de mots de passe forts, du SSH, etc.;
    \item la vigilance et la prévoyance, par exemple par la mise en place de
    sauvegardes avant et après chaque changements sur le serveur;
    \item le contrôle du bon fonctionnement de chaque élément.
\end{itemize}


\subsection{Sécurité}
\label{subsec:securite}

Du côté de la sécurité, nous avons quelques contraintes reprises ci-dessous :

\begin{itemize}
    \item mise en place d'une politique utilisateur;
    \item mise en place de quotas;
    \item partitionnement et gestion du disque \textit{(\textbf{LVM} et
    \textbf{RAID})};
    \item mise en place d'une stratégie de sauvegarde;
    \item désactivation des éléments inutiles et des mises à jours;
    \item mise en place d'un antivirus, d'un firewall, etc.
\end{itemize}

%%% Local Variables:
%%% mode: latex
%%% TeX-master: t
%%% End:
