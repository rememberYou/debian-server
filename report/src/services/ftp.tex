\subsection{FTP}
\label{subsec:ftp}

Un serveur FTP (\emph{File Transfer Protocol}), permet de transférer des
fichiers par Internet ou par le biais d'un réseau informatique local
(intranet). Pour ce serveur, il sera disponible au travers du réseau local.

\subsubsection{Choix du serveur}
\label{subsubsec:choix-serveur}

Pour un maximum de sécurité, VsFTPd (\emph{Very Secure FTP Daemon}) a été utilisé. \\
Ce serveur FTP est fortement axé sécurité : c'est l'un des premiers logiciels
serveurs à mettre en \oe{}uvre la séparation des privilèges, minimisant ainsi les
risques de piratage.

Dans sa configuration par défaut, VsFTPd est très restrictif :

\begin{itemize}
    \item seul le compte anonyme est autorisé à se connecter au serveur, et en
      lecture seule;

    \item les utilisateurs ne peuvent accéder qu'à leur compte.
\end{itemize}

\subsubsection{Configuration}
\label{subsubsec:config}

La totalité de l'implémentation se trouve dans les fichiers :

\begin{itemize}
\item \textit{/etc/vsftpd.conf}

\item \textit{/etc/pam.d/vsftpd}
\end{itemize}

Voici les différentes étapes et options qui ont été effectuées :
\begin{itemize}
    \item installation de vsFTPd

        \begin{lstlisting}[language=bash]
            # Installation of vsftpd.
            apt-get install -y vsftpd
        \end{lstlisting}

    \item choix du port et du monitoring

        \begin{lstlisting}[language=bash]
            # Change the number of port to transmit.
            listen_port=52152

            # Basic monitoring.
            setproctitle_enable=YES
        \end{lstlisting}

    \newpage
    \item configuration de vsFTPd ;

        \begin{lstlisting}[language=bash]
            # Set vsftpd in standalone mode.
            listen=YES
            # Block the not allowed users.
            anonymous_enable=NO
            # Allow the local connections
            local_enable=YES
            write_enable=YES
            local_umask=022
            # Allow connection for guests users.
            guest_enable=YES
            # Default user for connections.
            guest_username=apache
            # Avoid local users go to /root.
            chroot_local_user=YES
            # Send virtual users into the default folder.
            local_root=/srv/www/
            # PAM manages the authentifications of the system.
            # We can set a login and a password to all the systems.
            pam_service_name=vsftpd
            # Create a default folder for users.
            user_config_dir=/etc/vsftpd/vsftpd_conf_users
        \end{lstlisting}

    \item configuration de PAM ;

        \begin{lstlisting}[language=bash]
            # Create the configuration file.
            ################ PAM VSFTPD CONFIGURATION ###############

            # Authentification"
            auth required /lib/x86_64-linux-gnu/security/pam_userdb.so
                db=/etc/vsftpd/login
            account required /lib64/security/pam_userdb.so
                db=/etc/vsftpd/login
        \end{lstlisting}

\end{itemize}

%%% Local Variables:
%%% mode: latex
%%% TeX-master: t
%%% End:
