\subsection{FTP}
\label{subsec:ftp}

Un serveur \emph{FTP} (File Transfer Protocol, permet de transférer des
fichiers par Internet ou par le biais d'un réseau informatique local
(intranet). \\ Dans notre cas, il sera disponible au travers du réseau local.

\subsubsection{Choix du serveur}
\label{subsubsec:choix-serveur}

Pour un maximum de sécurité, \emph{vsFTPd} (Very Secure FTP Daemon) a été utilisé. \\
Ce serveur FTP est fortement axé sécurité : c'est l'un des premiers logiciels
serveurs à mettre en œuvre la séparation des privilèges, minimisant ainsi les
risques de piratage.

Dans sa configuration par défaut, VsFTPd est très restrictif :

\begin{itemize}
    \item Seul le compte anonyme est autorisé à se connecter au serveur, et en
      lecture seule;

    \item Les utilisateurs ne peuvent accéder qu'à leur compte.
\end{itemize}

\subsubsection{Configuration}
\label{subsubsec:config}

À terminer...

%%% Local Variables:
%%% mode: latex
%%% TeX-master: t
%%% End:
