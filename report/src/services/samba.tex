\subsection{Samba}
\label{subsec:samba}

\emph{Samba} est un outil permettant de partager des dossiers et des imprimantes
à travers un réseau local. Son utilisation est conseillée pour partager de
manière simple des ressources entre plusieurs ordinateurs \\ Il est compatible
avec les systèmes d'exploitation suivants : \emph{Windows}, \emph{macOS}, ainsi
que des systèmes \emph{GNU/Linux}, \emph{*BSD} et \emph{Solaris}} dans lesquels
une implémentation de Samba est installée.

\subsubsection{Configuration}
\label{subsubsec:configuration}

La configuration du serveur Samba se déroule en trois parties, mais tout
d'abord, il est nécessaire de créer le dossier de partage et de lui donner les
droits appropriés.

\begin{enumerate}
\item configuration de Samba \emph{(désignation du \textbf{workgroup}, choix
    du nom de \textbf{netbios}, etc.)} ;

\item configuration du partage pour le groupe \og \textit{users} \fg
  \emph{(désignation du chemin, des droits, etc.)} ;

\item configuration du partage du dossier \og \textit{home} \fg des utilisateurs
  \emph{(désignation des droits, vérification de l'identité, etc.)}.
\end{enumerate}

%%% Local Variables:
%%% mode: latex
%%% TeX-master: t
%%% End:
