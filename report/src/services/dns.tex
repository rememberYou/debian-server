\subsection{DNS}
\label{subsec:dns}

Le DNS (\emph{Domain Name System}), est un service permettant de  résoudre un nom
de domaine. \\
De fait, les serveurs étant identifiés par leur adresse IP, il a fallu créer un
processus afin d'associer leur adresse à un nom plus simple à retenir, le \og
nom de domaine \fg.

\subsubsection{Sélection du DNS}
\label{subsubsec:selection-dns}

Il a été choisi d'utiliser BIND9 (\emph{Berkeley Internet Name Daemon}). \\
Celui-ci est le serveur DNS le plus utilisé sur Internet, spécialement sur les
systèmes de type UNIX et est devenu de facto un standard.

\subsubsection{Mise en place}
\label{subsubsec:miseen-place}

La majorité de l'implémentation se trouve dans le fichier
\textit{/etc/bind/named.conf.options}.

Le DNS a été installé et configuré sur le serveur en différentes étapes :
\begin{itemize}
\item installation de BIND9;

  \begin{lstlisting}[language=bash]
    # Installation of bind9.
    apt-get install bind9 bind9utils bind9-doc -y
  \end{lstlisting}

\item création des ACL (\emph{Access Control List}) définissant le réseau local;

  \begin{lstlisting}[language=bash]
    acl goodclients {
      10.1.0.0/16;
      localhost;
      localnets;
    };
  \end{lstlisting}

  \newpage

\item création et configuration du serveur DNS en lui-même :
  \begin{itemize}
  \item[\tiny$\bullet$] acceptation des requêtes uniquement pour le réseau
    interne;

    \begin{lstlisting}[language=bash]
      recursion yes;
      allow-query { goodclients; };
    \end{lstlisting}

  \item[\tiny$\bullet$] configuration des forwarders;

    \begin{lstlisting}[language=bash]
      forwarders {
        8.8.8.8;
        8.8.4.4;
      };
      forward only;
    \end{lstlisting}

  \item[\tiny$\bullet$] activation de \emph{DNSSEC} qui sécurise les
    données envoyées par le DNS;

    \begin{lstlisting}[language=bash]
      dnssec-enable yes;
      dnssec-validation yes;
    \end{lstlisting}

  \item[\tiny$\bullet$] implémentation de la
    RFC1035\footnote{\url{http://abcdrfc.free.fr/rfc-vf/rfc1035.html}}
    \footnote{\url{http://www.bortzmeyer.org/1035.html}}.

    \begin{lstlisting}[language=bash]
      # Conform to RFC1035
      auth-nxdomain no;
      listen-on-v6 { any; };
    \end{lstlisting}
  \end{itemize}
\end{itemize}

%%% Local Variables:
%%% mode: latex
%%% TeX-master: t
%%% End:
