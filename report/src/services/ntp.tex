\subsection{NTP}
\label{subsec:ntp}

Le \textit{NTP} ({\emph{Network Time Protocol}), est le protocole utilisé afin
de synchroniser les machines du réseau local en fonction de l'horloge du
serveur.

\subsubsection{Principe}
\label{subsubsec:principe}

Bien que tout équipement informatique dispose d'une horloge, celle-ci
dérive comme toute montre ordinaire, ce qui peut amener a des erreurs de
synchronisation. \\
La nécessité de synchroniser des équipements en réseau paraît alors évidente.

Chaque machine peut être à la fois serveur et client.
Elle sélectionnera un serveur de temps dans sa configuration, et récupérera
l'heure, ainsi qu'un numéro de strate, \emph{n}, et se déclarera
de strate \emph{n + 1}.

L'architecture du réseau est en arborescence, et divisée en trois couches :

\begin{enumerate}
    \item les plus précises sources \textit{(horloges atomiques,
        récepteurs GPS, ...)} sont de \emph{strate 0};

    \item les serveurs diffusant l'heure des sources sont de \emph{strate 1};

    \item les serveurs de \emph{strate 2} sont généralement accessibles au public.
\end{enumerate}

\subsubsection{Configuration du serveur}
\label{subsubsec:configuration-serveur}

Voici les différentes étapes et options que nous avons effectuées :

\begin{itemize}
    \item activation des statistiques NTP;
    \item ajout de trois serveurs \emph{(un belge et deux européens)};
    \item activation de l'échange de l'heure avec tout le monde \emph{(aucune
modification n'est acceptée)};
    \item activation de la synchronisation avec les machines du réseau local.
\end{itemize}


\subsubsection{Configuration du client}
\label{subsubsec:configuration-client}

Sur le client, la configuration est beaucoup plus simple :

\begin{itemize}
    \item activation des statistiques NTP;
    \item ajout du serveur local.
\end{itemize}

%%% Local Variables:
%%% mode: latex
%%% TeX-master: t
%%% End:
