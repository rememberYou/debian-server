\subsection{NFS}
\label{subsec:nfs}

Le NFS (\emph{Network File System}), est un protocole qui permet à un ordinateur
d'accéder à des fichiers distants via un réseau. \\
Ce système de fichiers en réseau permet de partager des données principalement
entre systèmes UNIX.

\subsubsection{Constatation}
\label{subsubsec:constatation}

Avant de commencer, il est à remarquer que, quelle que soit sa version, NFS est
à déployer dans un réseau local et n'a pas de vocation à être ouvert sur
Internet. \\ En effet, les données qui circulent sur le réseau ne sont pas
chiffrées et les droits d'accès sont accordés en fonction de l'adresse IP du
client qui peut être usurpée.

\subsubsection{Configuration côté serveur}
\label{subsubsec:config-serveur}

La totalité de l'implémentation se trouve dans le fichier
\textit{/etc/exports}.

Voici les différentes étapes et options qui ont été effectuées :
\begin{itemize}
\item installation des différents services indispensables au NFS ;

  \begin{lstlisting}[language=bash]
    # Installation of NFS.
    apt-get install nfs-kernel-server nfs-common -y
  \end{lstlisting}

\item création du dossier de partage, et ajout de droits
  spécifiques ;

  \begin{lstlisting}[language=bash]
    mkdir /srv/share
    chmod 777 /srv/share
  \end{lstlisting}

\item activation du partage sur le réseau local et configuration dudit
  partage (autorise la lecture et l'écriture, retire les droit \textbf{root} à
  distance et désactivation de la vérification de sous-répertoires);

  \begin{lstlisting}[language=bash]
    /srv/share  192.168.0.0/16(rw,no_subtree_check,root_squash)
  \end{lstlisting}

\item mises à jour de la tables des systèmes de fichiers partagés.

  \begin{lstlisting}[language=bash]
    # Update the table of exported file systems.
    exportfs -av
  \end{lstlisting}
\end{itemize}

\subsubsection{Configuration côté client}
\label{subsubsec:config-client}

Sur le client, la configuration est similaire :

\begin{itemize}
\item installation des différents services indispensables au NFS;

  \begin{lstlisting}[language=bash]
    # Installation of NFS.
    apt-get install nfs-common -y
  \end{lstlisting}

    \item création du dossier de partage, et ajout de droits
      spécifiques;

  \begin{lstlisting}[language=bash]
    mkdir /mnt/share/users
    chmod 777 /mnt/share/users
  \end{lstlisting}

\item installation d'AutoFS;

  \begin{lstlisting}[language=bash]
    # Installation of AutoFS.
    apt-get install AutoFS
  \end{lstlisting}

\item configuration d'AutoFS

  Contenu du fichier \textit{/etc/auto.master} :

  \begin{lstlisting}[language=bash]
    /mnt/share    /etc/auto.nfs    --ghost,timeout=30
  \end{lstlisting}

  Contenu du fichier \textit{/etc/auto.nfs} :

  \begin{lstlisting}[language=bash]
    users -noexec,nosuid,rw,ghost \
    192.168.77.131:/srv/share/users
  \end{lstlisting}

  \underline{Remarque :} l'adresse IP ``192.168.77.131'' étant celle du serveur NFS.

  La configuration ci-dessus permet la création d'un point de montage lors de
  l'accès au répertoire.
\end{itemize}

%%% Local Variables:
%%% mode: latex
%%% TeX-master: t
%%% End:
