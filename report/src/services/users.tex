\subsection{Utilisateurs}
\label{subsec:users}

Ce n'est pas un service en tant que tel, mais, lors de la création d'un nouvel
utilisateur, celui-ci passe par plusieurs étapes :

\begin{itemize}
    \item il est ajouté à la base de données ;
    \item un site est créé au nom de l'utilisateur ;
    \item il est ajouté à la liste des utilisateurs de Samba ;
    \item il possède désormais un espace de stockage.
\end{itemize}

Lors de la suppression de l'utilisateur, il est nécessaire de supprimer ses
données.

\subsubsection{Configuration}
\label{subsubsec:configure}

\begin{itemize}
    \item création de l'utilisateur

        \begin{lstlisting}[language=bash]
            # Creates an user.
            useradd $username -m -G users -s /bin/bash
            echo -e "$password\n$password" | (passwd $username)

            quotatool -u $username -bq 400M -l 500M /home
            smbpasswd -a $username
            # Each user has access to the 'deepblue' database.
        \end{lstlisting}

    %\newpage
    \item création du site dans le fichier \textit{/etc/apache2/sites-enabled/\$username.lan.conf}

        \begin{lstlisting}[language=bash]
            # Creates a website for a specific user.
            <VirtualHost *:80>
                ServerAdmin $username@$username.lan
                ServerName $username.lan
                ServerAlias www.$username.lan
                DocumentRoot /srv/www/$username/www
                ErrorLog /srv/www/$username.lan/logs/error.log
                CustomLog /srv/www/$username.lan/logs/access.log
                    combined
            </VirtualHost>
            Alias /$username '/srv/www/$username/www'
            <Directory '/srv/www/$username/www'>
                Options Indexes FollowSymLinks MultiViews
                AllowOverride All
                Order deny,allow
                Allow from all
            </Directory>

            mkdir -p "/srv/www/$username/www"
            mkdir -p "/srv/www/$username.lan/logs/"
            chmod 755 "/srv/www/$username/www"
        \end{lstlisting}

    \item création de la page d'accueil du site \textit{/srv/www/\$username/www/index.html}

        \begin{lstlisting}[language=bash]
            <!DOCTYPE html>
            <html>
            	<head>
                	<meta charset="utf-8">
                	<title>Server page</title>
                </head>
                <body>
                    <div id="greetings">
                    	<h1>Welcome \$username!</h1>
                        <h2>This is your own server page.</h2>
                    </div>
            	</body>
            </html>
        \end{lstlisting}

\end{itemize}


%%% Local Variables:
%%% mode: latex
%%% TeX-master: t
%%% End:
