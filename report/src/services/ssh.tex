\subsection{SSH}
\label{subsec:ssh}

Le SSH (\emph{Secure Shell}), est un protocole de communication
sécurisé. Celui-ci impose un échange de clés de chiffrement en début de
connexion.

\subsubsection{Type d'authentification}
\label{subsubsec:type-authentification}

Il existe plusieurs manières de s'authentifier sur le serveur via SSH.

Les deux authentifications les plus utilisées sont :
\begin{enumerate}
\item par mot de passe ;
\item par clés publiques et privées du client.
\end{enumerate}

L'identification automatique par clés a été mise en place pour ce serveur. De ce
fait, il n'est plus nécessaire d'entrer le mot de passe à chaque connexion à
distance. \\
Cette méthode est plus complexe à mettre en place, mais elle est surtout plus
pratique.

On remarque rapidement son utilité si on se connecte fréquemment au serveur, car
plus aucun mot de passe n'est demandé.

\subsubsection{Implémentation}
\label{subsubsec:implementation}

La majorité de l'implémentation se trouve dans le fichier
\textit{/etc/ssh/sshd\_config}. \\
Le serveur a été configuré en respectant ces critères :
\begin{itemize}
\item installation de \textit{openssh} ;

  \begin{lstlisting}[language=bash]
    # Installation of OpenSSH.
    apt-get install openssh-server -y
  \end{lstlisting}

\item changement de port et passage à la version 2 de SSH pour plus de sécurité;

  \begin{lstlisting}[language=bash]
    # Using port number 62000
    Port 62000
    # Using Protocol 2 of SSH.
    Protocol 2
  \end{lstlisting}

  \newpage

\item ajout d'une bannière (disponible dans le fichier \textit{/etc/ssh-banner/banner});

  \begin{lstlisting}[language=bash]
    The debianThink server is for authorized personnel only.
    WARNING! Access to this device is restricted to those
    individuals with specific permissions. If you are not an
    authorized user, disconnect now.  Any attempts to gain
    unauthorized access will be prosecuted to the fullest
    extent of the law.

    All access and use may (not will) be monitored
    and/or recorded.
  \end{lstlisting}

\item connexion en tant que \textbf{root} ;

  \begin{lstlisting}[language=bash]
    # Privilege separation for security.
    UsePrivilegeSeparation yes
  \end{lstlisting}

\item déconnexion après 120 secondes d'inactivité ;

  \begin{lstlisting}[language=bash]
    # Deactivation of the login in root and disconnection
    # after 120 seconds if no connections.
    LoginGraceTime 120
    PermitRootLogin no
    StrictModes yes
  \end{lstlisting}

\item désactivation de la connexion par mot de passe, vu que l'authentification
  passe par les clés RSA.

  \begin{lstlisting}[language=bash]
    # We deny the authentication by password.
    PasswordAuthentication no
  \end{lstlisting}
\end{itemize}

Ensuite, une génération et un chiffrement d'une paire de clés publique / privée
sur la machine cliente ont été nécessaires.

\begin{lstlisting}[language=bash]
  ssh-keygen -t rsa -b 4096 -C $email -f "/$USER/.ssh/id_rsa" \
  -N ""
\end{lstlisting}

Une fois cela fait, la clé publique de celle-ci a été enregistrée sur le serveur
dans le fichier \\
\textit{/etc/ssh/ssh\_host\_rsa\_key} afin d'accepter sa connection au serveur.

%%% Local Variables:
%%% mode: latex
%%% TeX-master: t
%%% End:
